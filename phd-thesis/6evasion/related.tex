\section{Related Works}
\label{sec:related}
A couple of previous research works focus on evading the Android sandbox or Android anti-virus scanners. Huang, et al.~\cite{huang2015towards} discovered two generic evasions that can completely evade the signature based on-device Android AV scanners, while we are focusing on the Android sandboxes used offline. Sand-Finger~\cite{Sand-Finger} collects fingerprints from 10 Android sandboxes and AV scanners to bypass current AV engines and all of fingerprints used by Sand-Finger are hardware-related or system-related, which are different from our user profile based  fingerprints. Timothy~\cite{Vidas} presented several sandbox evasions by analyzing the differences in behavior, performance, hardware and software components. He also revealed that dynamic analysis platforms for malware that purely rely on emulation or virtualization face fundamental limitations that may make evasion possible. The above approaches are mainly related to detect the Android emulator environment. Wenrui, et al.~\cite{Diao} proposed to evade the Android runtime analysis through by identifying automated UI explorations. Similarly, our work is to distinguish the difference between the sandbox environment and the real user device environment through usage-profile based fingerprints. 

In \cite{blackthorne2016avleak} Blacktorne et al. proposed \textit{AVLeak} a blackbox technique to extract emulator fingerprints. Although it address the similar problem about how to efficiently extract emulator fingerprints the implemented methodology is not suitable on Android. Moreover \textit{AVLeak} was not focused on usage-profile based fingerprints which are quite relevant ones for the mobile ecosystem.
