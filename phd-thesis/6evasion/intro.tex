\chapter{Identifying and Evading Android Sandbox through Usage-Profile based Fingerprints}

% outline
% 1) the Android device and apps are very popular, as well as the growth of Android malwares.
% 2) Static analysis and dynamic analysis. say that the dynamic analysis is much better than the static analysis.
% 3) New Evading Techniques through the fingerprints of sandboxes 
% 4) 

Among the massive volume of Android apps used by Android users, there exists a lot of Android malware, which become the main threat for Android users currently. To mitigate the threat of the Android malware, static and dynamic analysis techniques are the main solutions to detect Android malware. Static analysis has the limitation on detecting the malware when using the code obfuscation, native code, Java reflection and packer. But, dynamic analysis can help detect such Android malware more precisely in its dynamic sandboxes. The traditional dynamic analysis sandbox is built either on the Android emulator or the real device to enable fast and effective malware detection. 

To evade dynamic analysis, some anti-emulator techniques~\cite{Petsas,Vidas,Jing} were proposed, and they are commonly used by Android malware. In general, these techniques were designed to obtain fingerprints of the runtime environment of the Android emulator. Recently, BareDroid~\cite{Mutti:2015:BLA:2818000.2818036} was proposed to use real devices to build the Android sandbox. This method can mitigate the anti-emulator techniques
, but we believe the arms race between dynamic analysis techniques and evasion techniques is endless. No matter whether the Android sandbox is built on the Android emulator or the real device, the Android malware can still evade the detection of the sandbox through identifying the difference between the emulated phone and the real user phone. 

In this chapter, we conduct a measurement on collecting fingerprints from public Android sandboxes, including AV sandboxes, online detection sandboxes and even sandboxes used by app markets. Through analyzing collected fingerprints, we propose an evading technique based on a new type of fingerprints of Android sandboxes: \textit{usage-profile based fingerprints} (e.g., the contact list information, SMS, and installed apps). The measurement result shows that these Android sandboxes have no usage-profile based fingerprints, or only have a fixed usage-profile based fingerprint, or have a random artifact fingerprint. So, most current Android sandboxes have not protected their usage-profile fingerprints, and are potentially bypassed by malware samples.

Therefore, by analyzing the usage-profile based fingerprints of the Android sandboxes and real Android user's device, the Android malware can still potentially evade the dynamic sandbox because of two reasons: 1) extracting some fingerprints, such as installed apps, are not very sensitive towards the verdict making of dynamic analysis, since those behaviors are commonly existed in benign apps. For example, Android Ads SDK extracts the installed app list for accurately distributing ads; 2) even though extracting some fingerprints (e.g. the contact list and SMS) may be regarded by dynamic analysis as malicious, the Android malware can still evade the detection by mimicking or repackaging~\cite{zhang2014viewdroid} as a contact app or SMS app. So, the advanced Android malware could firstly inspect these fingerprints and then launch other more powerful behaviors (e.g., rooting and sending SMS) if it does not identify the current environment as a sandbox environment.

To summarize, this work makes the following contributions:
\begin{itemize}
\item 1) {\bf New problem.} We propose a new Android sandbox fingerprinting technique, which is based on the careless design of usage-profiles in most current sandboxes. We observe that malware developers can collect usage-profile based fingerprints from many Android sandboxes and then leverage these fingerprints to build a generic sandbox fingerprinting scheme for the sandbox analysis evasion. 
\item 2) {\bf Implementation.} We conduct a measurement on collecting usage-profile based fingerprints on popular Android sandboxes. The results show that most Android sandboxes designers have not protected these fingerprints by generating the random fingerprints every time for running a different sample. Only few sandboxes generate the random fingerprints, but these random fingerprints are different from fingerprints in user's real phones.
\item 3) {\bf Mitigations.} We propose mitigations to further guide a proper design of these sandboxes against this hazard.
\end{itemize}

The remainder of the paper is structured as follows: in Section \ref{sec:back} we introduce background and motivations underlying our research, then in Section \ref{sec:design} we discuss our system design and in Section \ref{sec:res} we present collected results which effectively shows the effectiveness of our techniques. Proposed mitigations and related work are discussed in Section \ref{sec:defense} and Section \ref{sec:related} respectively. Finally, Section \ref{sec:end} concludes the paper.